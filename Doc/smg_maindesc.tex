\hyperbaseurl{http://smg.sf.net/}

%\newcommand{\newfile}[1]{\texttt{#1}}
\newcommand{\file}[1]{\lstinline[basicstyle=\ttfamily]!#1!\xspace}
\newcommand{\code}[1]{\lstinline[basicstyle=\ttfamily]!#1!\xspace}
\newcommand{\listcode}[1]{\lstinline[basicstyle=\ttfamily\footnotesize,indent=35pt]!#1!\xspace}
%language=Pascal prevents funny typesetting when # characters are used.
\newcommand{\incode}[1]{\lstinline[basicstyle=\ttfamily,language=Pascal]!#1!\xspace}
\newcommand{\smg}[1]{\lstinline[basicstyle=\ttfamily,language=Pascal]!#1!\xspace}
\newcommand{\SMG}[1]{\smg{SMG_#1}}
\newcommand{\SM}[1]{\smg{SM_#1}}

\newcommand{\INITSTATE}{\hyperlink{INITSTATE}{\smg{INIT_STATE}}\xspace}
\newcommand{\SMOBJ}{\hyperlink{SMOBJ}{\SM{OBJ}}\xspace}
\newcommand{\SMEVT}{\hyperlink{SMEVT}{\SM{EVT}}\xspace}
\newcommand{\SMNAME}{\hyperlink{SMNAME}{\SM{NAME}}\xspace}
\newcommand{\STATE}{\hyperlink{STATE}{\smg{STATE}}\xspace}
\newcommand{\EVENT}{\hyperlink{EVENT}{\smg{EVENT}}\xspace}
\newcommand{\TRANS}{\hyperlink{TRANS}{\smg{TRANS}}\xspace}
\newcommand{\TRANSeq}{\hyperlink{TRANSeq}{\smg{TRANS=}}\xspace}
\newcommand{\TRANSplus}{\hyperlink{TRANSplus}{\smg{TRANS+}}\xspace}
\newcommand{\CODE}{\hyperlink{CODE}{\smg{CODE}}\xspace}
\newcommand{\CODEbegin}{\hyperlink{CODEbegin}{\smg{CODE_\{}}\xspace}
\newcommand{\CODEend}{\hyperlink{CODEend}{\smg{CODE_\}}\xspace}}
\newcommand{\PROMELAbegin}{\hyperlink{PROMELAbegin}{\smg{PROMELA_\{}}\xspace}
\newcommand{\PROMELAend}{\hyperlink{PROMELAend}{\smg{PROMELA_\}}\xspace}}
\newcommand{\SMDEF}{\hyperlink{SMDEF}{\SM{DEF}}\xspace}
\newcommand{\SMIF}{\hyperlink{SMIF}{\SM{IF}}\xspace}
\newcommand{\SMELSE}{\hyperlink{SMELSE}{\SM{ELSE}}\xspace}
\newcommand{\SMEND}{\hyperlink{SMEND}{\SM{END}}\xspace}
\newcommand{\SMTRACE}{\hyperlink{SMTRACE}{\SM{TRACE}}\xspace}
\newcommand{\SMINCL}{\hyperlink{SMINCL}{\SM{INCL}}\xspace}
\newcommand{\SMDESC}{\hyperlink{SMDESC}{\SM{DESC}}\xspace}

\begin{center}
{\itshape Copyright \copyright 2001-2003}
\end{center}

\begin{abstract}
  The SMG utility can be used to scan an input file for specific
  directives that describe a State Machine (States, Events,
  Transitions, and associated Code segments) and generate several
  different outputs: C code to implement that State Machine, Promela
  code to implement a formal verification of the State Machine using
  \hyperlink{Spin}{Spin}, and a graphical representation of that State
  Machine for analytical purposes.  In mechanical terms, the SMG may
  be thought of as a specific-purpose C preprocessor.

  The SMG utility may be obtained from: \code{http://smg.sf.net/}

\end{abstract}

\tableofcontents

%\part{?}

\section{Background}

\subsection{Tunnel Vision}

Modern programming techniques are often oriented around a procedural
language such as C or a similarly structured object-oriented
implementation.  When developing software using these languages, the
prevalent coding methodology is to design a set of instructions and
functions that will perform the intended sequence of tasks, using a
top-down approach.  This sequence of tasks is identified as the
primary objective of the program being developed.  This programming
methodology is referred to as {\bf task-oriented} programming.

Frequently, however, the attention to developing support for portions
of the code not related to the primary objective is diminished or in
some cases non-existent.  This results in unhandled or badly handled
error paths and an inflexibility of the code regarding changes or
deviations in input or operating environment.  The top-down approach
can exacerbate this issue as the focus narrows as details are
introduced, thereby causing those details to be developed with
primarily local concerns.

Additionally complicating this programming methodology is the attempt
to introduce multi-threaded scheduling to improve the overall
responsiveness of the program to multiple simultaneous inputs.  When
introducing threading significant attention must be provided to
properly protecting common data structures and maintaining each thread
relative to all other threads and the overall program state.

Finally, validation of the resulting code is left to a rote process of
verifying functionality and output of the code when presented with
sample sets of input; coverage tools attempt to verify that most of
the written code has been executed, but this does not necessarily mean
that the full range of input or sequencing has been tested in the
process.  A technique known as modelling and formal verification can
be used to abstract the functionality of the code, but this is usually
only used in the architectural and early design phases; the
resemblance of the final code to the formal verification model is
tenuous at best and often minimal.  Developers tend to handle
unexpected input or events as ``bugs'' by providing local fixes to
ensure that their code does not fail when these occur, but without
consideration of how the system as a whole should properly handle
those situations and rarely by returning to and updating the formal
verification model.

\subsection{Asynchronous Event Handling}

Another one of the frequent challenges in developing computer software
is implementing and maintaining code that can fully manage an
operation that is subject to asynchronous events.  Asynchronous events
can take many forms, including: user-input, operation completion
notifications, interrupts, and operation requests.  In general, an
asynchronous event can occur at any time and the software must be
capable of determining what the appropriate response to that event is
at that point in time.

These event-oriented environments are found in most non-computation
oriented code and are especially prevalent in: Internet Servers,
Device Drivers, and GUI implementations.

The task-oriented coding style discussed previously makes it difficult
to anticipate and handle exceptions to the sequence of tasks which are
introduced by Asynchronous Events and also leads to assumptions
regarding when events will occur and what types of events will occur
at various delivery points.  Often ignored or lightly-addressed is the
importance of determining the appropriate response to {\em all} types
of events that may occur at {\em each} point where they may occur.

One technique commonly used in applications written for an
Asynchronous Event environment is to declare an event queue onto which
all events are placed in order of occurrence.  The code then removes
an event from the head of the queue, processes that event in a
task-oriented manner until the event has been completely handled, and
then returns to the queue to obtain the next event.  This algorithm
may be useful in some circumstances but its appropriateness is often
invalid when events cannot be queued, when events must be prioritized,
and when some events may interrupt other events.

\subsubsection{EDSM Methodology}

An alternative coding style used in these situations is the {\bf
  Event-Driven State Machine} or {\bf EDSM} methodology\footnote{The
  State Machine described in the document is a Finite Deterministic
  Mealy Machine, also referred to as a Deterministic Finite Automata
  or DFA.}.  In this methodology, events are typically delivered to a
common entry point and then a specific function or switch statement is
invoked based on the current state or the event.  Once in that code
has been invoked, the other parameter (entry or state) is examined to
select the code to execute for that state/event combination.  While
this style is more flexible in terms of handling unexpected events it
is more arduous to develop code in this style due to the mechanics of
reproducing state and event selection code throughout the code and the
loss of the ``flow'' perspective for the primary sequence of tasks.

Some software designs do implement an EDSM methodology, but
unfortunately, most of the software in these implementations focuses
on the expected sequence of operations only; it's frequently the case
that not all events are handled at each event-delivery juncture in the
code.

Furthermore, in situations where state machines are implemented in the
software, the completeness of the state machine tends to
logarithmically increase the complexity and obscurity of that state
machine, making them hard to understand and maintain.  Each possible
state has to consider the potential occurrence of an ever-increasing
number of events, and the addition of each state causes an
event-number of new paths through the state machine.

This type of code is difficult to maintain, especially for developers
introduced to the code after it is written.  State machine code is
often spread throughout the body of the main code, making it hard to
understand the entirety of the state machine.  Furthermore, the state
machine code has significant side effects; any change to a state
machine's structure (i.e. adding a new state or changing the response
to an event) will significantly impact code executing to handle future
events.  Understanding, predicting, and assessing the validity of any
changes to the state machine quickly becomes a monumental task as the
size of the state machine grows.

A formal verification model is a highly useful tool to properly
evaluate and manage additions (states or events) to this type of code,
but again, since the formal verification model is often retired or
divergent at this point in development the opportunity is not often
available.  Some institutions even separate the coding and formal
verification modelling into separate groups, the former being handled
by architects and the latter by developers.  Sometimes the developers
are not even aware of the existence of formal verification models.

\subsection{Non-blocking Operations}

Another more recent advancement facing many implementations is the
introduction of non-blocking function calls for performing various
asynchronous tasks such as: I/O operations and remote procedure calls
(RPC) found in distributed environments.  In these situations, the
program's initial request is not completed when the request call
completes; instead the request is processed in parallel or at some
later point in time and when completed, the initial program is
notified of that completion (usually by invocation of a callback
function).

This complicates the normally task-oriented methodology in the
following ways:
\begin{enumerate}
\item The task-oriented flow is broken up into several different
  ``chunks'' of code, split by the need to await a completion
  indication after performing a non-blocking request.
\item The potential for other events to occur while waiting for or
  instead of the requested operation, with those events being
  delivered before or instead of the primary operation's completion
  indication.
\item The potential for request re-entrancy, allowing the program to
  handle a new request while waiting for an interim non-blocking
  operation to complete for a previous request.  While this tends to
  increase the overall efficiency of the program, it significantly
  affects the management of common data and the management of this
  now-pipelined implementation.
\item The potential for the non-blocking request to signal completion
  at any point after the request call is made, possibly even before
  the request call returns to the calling process.
\end{enumerate}
These complications require significant additional care in properly
and efficiently implementing applications for this type of environment.


\section{The SMG Solution}

The proposed approach to handling these issues is to attempt to
reconcile the EDSM methodology with the more customary task-oriented
programming styles.  In order to do this, we seek to automate the
mechanics of EDSM and re-introduce the conceptual perspective of
task-oriented coding styles.  This is done by beginning with the EDSM
methodology and making the following set of observations and changes.

To facilitate this approach, the SMG tool has been developed.  Using
the SMG tool, the developer describes the state machine by a set of
SMG directives interspersed with the program's normal C code.  

The SMG directives are intended to be more succinct in describing the
state machine than the corresponding C code, thereby allowing the
state machine to be more easily recognized and understood even at the
input specification level.  The SMG directives also allow default
functionality to be specified and allows the state machine to be more
``naturally'' described in parallel to the task-oriented code
segments.

Once the code (including SMG directives) has been developed, the code
is passed through the State Machine Generator (SMG) as a preprocessing
stage.  The SMG utility converts the SMG directives into C code which
implements the described state machine, allowing the result to be
passed to the C compiler as a combination of the state machine and the
task-oriented code, providing a complete functional solution.

\subsubsection{Visual Analysis}

The SMG utility also produces a description of the state machine that
may be passed to the \hyperlink{GraphViz}{GraphViz} utility to obtain a
graphical representation of the state machine.  The graphical
representation allows easy interpretation and maintenance of the state
machine.

The graphical representation shows the various states as nodes on the
graphs and labels the arcs that connect those nodes with the names of
the events that cause that transition.  The arc labels also indicate
which code objects are executed before and after the event causes the
actual state to change.

Expected transitions are represented by thick arcs, making it easy to
follow the normal code flow and differentiated from unusual or error
handling transitions.

SMG will also {\em automatically} generate an error state if there are
any states wherein the result of an event is not defined explicitly by
SMG directives.  The corresponding C code generated will cause a
runtime error call to a programmer-supplied error routine if this
undefined event transition occurs).  All transitions to this error
state will be labelled as errors for easy analysis; final versions of
the code should not contain any undefined transitions of this type.

SMG will also group states in order to simplify the diagram.  For
example, if a group of 5 states have normal transitions to other
states, but they all transition to an error state when a particular
event occurs, SMG will assign those 5 states to a group and separately
indicate that the group transitions to the error state on that event.
This grouping is performed automatically.

\subsubsection{Functional Verification}

The third SMG output is a \hyperlink{SPIN}{SPIN}/Promela model of the
state machine.  This model is automatically extracted from the states,
events, and transitions defined by the SMG directives.  It is
therefore capable of describing the overall functionality of the
system for SPIN functional verification without any additional input
from the developer.

To supplement and refine the functional verification process, standard
Promela code can be identified in the input file by bracketing SMG
directives.  The SMG utility will then direct this Promela code into
the appropriate locations of the generated model as a parallel to the
C code for the same transitions.  In this way, both the overall
functionality and detailed modelling code can be maintained in the SMG
input file along with the corresponding C code.  Thus, when additional
states or events must be considered, or when one or more transitions
must be redefined, the Promela model is correspondingly adjusted for
continued formal verification.

As with the graphical and C outputs, the emitted Promela code
generates assertions for undefined transitions, causing them to be
flagged during formal verification (or they can be disabled to
validate the currently defined subset of the application).


\subsection{SMG Development Cycle}

SMG development is done by designing C code modules as in a normal
development operation and implementing SMG directives in one or more
of the modules.  Modules containing SMG directives typically have a
``\file{.sm}'' suffix to distinguish them from normal ``\file{.c}''
files (and so that SMG can generate a \file{.c} from the \file{.sm}).
The SMG directives can appear exlusively in a file themselves or then
can be intermixed throughout the C code to the degree desired by the
coding style being used.

The SMG directives also establish the {\bf States} and {\bf Events}
that can occur, and optionally any {\bf entry points} into the code
that represent Events.  The definitions of these States, Events, and
entry points are output by SMG into a header include file
(\file{xxx_smdefs.h}).  All modules needing to use a State or Event
value or make other references to the state machine can include this
file to obtain the appropriate external definitions.

Once the C code and SMG directives have been written, the {\bf\tt smg}
utility should be run on the \file{.sm} input files.  The graphical
state machine output should be examined to visually verify the
implementation, followed by formal verification using SPIN and the
output Promela model.  Once the state machine has been formally
verified, the output C code should be compiled by the C compiler and
the linker for all C source files (some of which may have been output
by the smg utility from \file{.sm} files).  All errors during the
development cycle are resolved in the input \file{.sm} file rather
than making modifications to interim files; the resulting SMG model is
therefore fully complete and functional.

\begin{figure*}
\input{smg_fig1.tex}
\end{figure*}
%\input{old_smg_fig1.tex}

\subsection{C Data Structures and Functions}

In order to implement the EDSM methodology, SMG attempts to make
minimal assumptions about the programming environment in which it will
generate the state machine.

SMG will automatically provide C definitions for the states and events
described in its directives.  There are only two required C data types
and based on the external interface for the program, and one required
C function (the error handling function).  The programmer can
optionally define the syntax of event delivery points within the code
with additional directives and C declarations, but this is not
required.  SMG does assume that if the event delivery points are not
specified that the C program will provide the event-specific interface
and invoke the state machine with an indication of the underlying
event type.

The first required C data type must be a structure and is the \SMOBJ
structure.  This is assumed to be an instance of a structure that is
global to the current task.  This structure may contain anything that
the programmer desires to implement but it must also contain the
following field: \code{<SM_NAME>_state_t sm_state;} where
\code{<SM_NAME>} is the declared name of the state
machine.\footnote{Multiple state machines may be defined in the same
  \file{.sm} file, and they may even share a common \SMOBJ data
  structure; the use of \code{<SM_NAME>} in all relavent state machine
  interfaces keeps the state machines distinct and separate.}  It is
recommended that the SMG-related code not refer to any global writable
data outside of the \SMOBJ structure although it is possible to
deviate from this under certain circumstances.

The second required C data type is the \SMEVT type.  This data type is
assumed to represent any and all information related to the current
event-initiated task.  A common implementation is for the programmer
to maintain a pool of \SMEVT structures which are used
one-by-one as primary events arrive.  The \SMEVT type may be
any valid C data type; it is used to maintain the task-oriented
context for the series of secondary events that are initiated by the
primary event.  The SMG code will not use the \SMEVT variable
directly but will pass the value of this variable from the
event-initiated state machine entry to the selected state/event
handling code.\footnote{Because the \SMEVT variable is not actually
  used by SMG-generated code, the associated variable may be unused
  throughout but it still must be a valid data type.}

%KWQ: this is really confusing.  Need a picture/example.

%Rework/clarify this section and combine/subsection the development cycle

%Re-org the following: SMG directive stuff, supporting code stuff


\section{SMG Directives Overview}

SMG directives are designed for simplicity in both parsing and
specification.  Although the user can choose to design a preprocessing
stage to preceed the SMG operation ({\it e.g.} m4), SMG directives are
simply presented and designed to be unambiguously distinguishable from
the C code into which they are placed.


\subsection{Common SMG Directives}

The list of directives allows for significant flexibility in how the
state machine is described and where the various operations occur.
Some of these capabilities are only needed for special situations, and
some are dependent on the way in which the state machine is
implemented.  Only the most basic directives are needed to identify a
state machine and all other directives may be introduced only as
needed when the basic directives are insufficient.  The basic
directives are listed below:

\begin{tabular}[h]{lll}
\\
\SMNAME & \TRANS \\
\SMOBJ & \SMEVT \\
\CODE & \CODEbegin & \CODEend \\
\end{tabular}

The reader is advised to attend to this subset of directives initially
and to review the additional keywords presented in the SMG Guide once
basic familiarity with SMG is attained.

%%% Local Variables: 
%%% mode: latex
%%% TeX-master: "smg_guide"
%%% End: 
