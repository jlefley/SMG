
\subsection{SMG Directive Keywords}

This section describes {\em all} supported SMG keywords.  As noted
previously, the initial review should focus on the subset of common
keywords, leaving the remaining keywords for more esoteric evaluation.

The syntax of SMG Directives is described in Figure~\ref{fig:syntax}.
In the figure, \smg{<xxx>} represents a user-supplied word, and
\smg{<xxx>...}  represents one or more user-supplied words.  Square
braces ('[' and ']') enclose optional portions of the directive and
are not actually present in the directive.  There are no restrictions
on \smg{<xxx>} words except those noted here and excepting whitespace.
Some \smg{<xxx>} words must be valid C identifiers:
\smg{<state-machine-name>}, \smg{<state-name>}, and
\smg{<event-name>}.

All SMG directives are line oriented, must be contained on a single
line, and must start with the SMG directive keyword at the first
character on the line.  The remainder of the line contains the
arguments and values for the SMG directive as whitespace separated
words, up to the end of the line; SMG directives do not support
line-continuation.

\begin{figure*}[p]
\begin{verbatim}

SM_NAME <state-machine-name>

SM_DESC <description>...

SM_OBJ  <C-type>...

SM_EVT  <C-type>...

SM_INCL <filename>

SM_DEF  <context-name>
SM_IF   <context-name>
SM_ELSE <context-name>
SM_END  <context-name>

STATE   <state-name> [ <description>... ]

INIT_STATE <state-name> [ <description>... ]

ST_DESC <description>...

EVENT <event-name> [ <entry-code-tag> <setup-code-tag> [ <description...> ] ]

EV_DESC <description>...

TRANS  <current-state> <event> <new-state> [ [<pre-code>] <post-code> ]
TRANS+ <current-state> <event> <new-state> [ [<pre-code>] <post-code> ]
TRANS= <current-state> <event> <new-state> [ [<pre-code>] <post-code> ]

## <comment>...

CODE    <code-tag> <code>...

CODE_{  <code-tag>
CODE_}

PROMELA_{      <code-tag>|HEADER|INIT
PROMELA_}

\end{verbatim}
\caption{SMG Directive Syntax}
\label{fig:syntax}
\end{figure*}


%%% Local Variables: 
%%% mode: latex
%%% TeX-master: "smg_guide"
%%% End: 


\begin{description}
  
\item[\hypertarget{SMNAME}{\SM{NAME}}] --- Specifies the start of a
  State Machine of the specified name.  An input \file{.sm} file(set)
  may contain multiple state machines.  The \SM{NAME} identifies the
  state machine to which all of the following state machine directives
  apply until a new \SM{NAME} state machine declaration is read.
  
\item[\hypertarget{SMDESC}{\SM{DESC}}] --- Descriptive text describing
  the purpose of the State Machine
  
\item[\hypertarget{SMOBJ}{\SM{OBJ}}] --- Type of the C object
  representing the instantiation of the State Machine.  This object
  persists across the lifetime of the state machine and typically
  contains user information needed by the code containing the state
  machine and must include an \code{sm_state} field used by the
  SMG-output code to maintain the current state of the code.
  
\item[\hypertarget{SMEVT}{\SM{EVT}}] --- Type of the C object
  associated with an event.  All information relative to the current
  event (except the event code itself) must be contained in this
  object. The body of the SMG-generated state machine will only have
  access to the \SMOBJ and \SM{EVT} variables.
  
\item[\hypertarget{SMINCL}{\SM{INCL}}] --- Includes another file.
  This operates in a similar manner to the \incode{#include}
  preprocessor directive in C except that it works during the SMG
  preprocessing stage rather than the C preprocessing stage.
  Additionally, the optional second parameter specifies the version of
  the file to be included.  The \SMINCL directive will look for the
  specified file based on a search path.  The search path looks in the
  following locations in the order specified:
  \begin{enumerate}
  \item The current directory
  \item \file{/usr/local/lib/SMG/*-vN}
    
    This location is searched only if the optional second argument is
    specified on the \SMINCL line specifying the version of the
    interface.  The \file{N} in the above path is replaced with the
    version number specified in the second argument and the \file{*}
    specifies a wildcard of all entries.  Thus, if the optional
    version specified is ``23'', the \SMINCL will search in all of the
    /usr/local/lib/*-v23 directories.
  \item \file{/usr/local/lib/SMG/*}
    
    All subdirectories of the above are checked.
  \item \file{/usr/local/lib/SMG}
  \end{enumerate}
  
\item[\hypertarget{SMDEF}{\SM{DEF}}] --- Defines a directive for use
  in the \SMIF/\SMELSE/\SMEND statements.  The \SM{DEF} is roughly
  equivalent to a C statement like the directive \incode{#define
    <context-name>}, except that the SM versions are parsed by SMG and
  \incode{#defines} are parsed by the C preprocessing stage of the
  compiler, which follows SMG parsing.  One way to negate this is to
  make it an SMG comment (see below).  This is a global directive and
  is not related to any specific state machine or other context.
  
\item[\hypertarget{SMIF}{\SM{IF}}] --- Specifies the beginning of a
  block of code that is only included if the \SMDEF statement for the
  corresponding \smg{<context-name>} has been seen.  This is a global
  directive and is not related to any specific state machine or other
  context.  Blocks of code demarcated by \SMIF/\SMEND statements may
  be nested, even within the same \smg{<context-name>}.  Each \SMEND
  statement exits the corresponding \SMIF context but no others.
  Context nesting is allowed but scope must be maintained.
  
\item[\hypertarget{SMELSE}{\SM{ELSE}}] --- Specifies the beginning of
  an alternate block of code that is only included if the \SMDEF
  statement for the corresponding \smg{<context-name>} has {\em not}
  been seen.

\item[\hypertarget{SMEND}{\SM{END}}] --- Specifies the end of a block
  of statements (C and SMG) that was previously identified with the
  \SMIF statement.
  
\item[\hypertarget{STATE}{\smg{STATE}}] --- Defines a state and
  optionally a description of that state.  This directive is optional;
  states are deduced by the SMG from \TRANS directives as needed.
  
\item[\hypertarget{INITSTATE}{\smg{INIT_STATE}}] --- Specifies the
  initialization for a state machine that is initialized at
  instantiation.  There may only be one of these for a state machine
  although it may appear in place of or with a corresponding
  \STATE directive for the same state.  State machines
  instantiated in this way {\em MUST} have zeroed contents when
  instantiated; since instantiation is external to the SMG generated
  code, the developer must insure that the new state machine, however
  created, is zero'ed (i.e. \code{memset(sm_obj, 0, sizeof(sm_obj))})
  before any events are delivered to that state machine.
  
  State machines which are not initialized at instantiation must be
  explicitly initialized with the \code{<name>_State_Machine_Init}
  operation before any events are delivered to that state machine.
  
\item[\hypertarget{STDESC}{\smg{ST_DESC}}] --- Provides additional
  description for the most recently declared \STATE.
  
\item[\hypertarget{EVENT}{\smg{EVENT}}] --- Defines an event, optional
  entry point declaration, optional input variable preprocessing, and
  an optional description of the event.  As with the \STATE
  directive, the \smg{EVENT} directive is optional if no entry point
  or preprocessing must be defined and events will be deduced as
  needed.
  
  Most uses of the \smg{EVENT} keyword will specify only the
  \smg{<event_name>}.
  
  In situations where an event entry point is automatically defined
  (e.g. an \hypertarget{eventlib}{Event Library}), the optional
  \smg{<entry-code-tag>} and \smg{<setup-code-tag>} specify the code
  to be generated for the event handling entry point.  The
  \smg{<setup-code-tag>} should not be defined ({\it i.e.} not
  \smg{--}) unless the \smg{<entry-code-tag>} is also defined (in
  other words, if \smg{<entry-code-tag>} is ``\smg{--}'' then
  \smg{<setup-code-tag>} must also be ``\smg{--}'').
  
\item[\hypertarget{EVDESC}{\smg{EV_DESC}}] --- Provides additional
  description for the most recently declared \EVENT.

\item[\hypertarget{TRANS}{\smg{TRANS}}] --- Defines a transaction.
  Specifies the handling of an Event for a specific current state in
  terms of the new state to go to when the event is received, along
  with any code to execute either before or after moving to the new
  state.  The \smg{TRANS} statement is the principle specification for
  SMG input.
  
  The \smg{<new-state>} may be ``\smg{--}'' to indicate that there is
  no state change associated with this event in the current state.
  
  The \smg{<pre-code-tag>} and/or the \smg{<post-code-tag>} elements
  may be ``\smg{--}'' to indicate that there is no pre-state-change or
  post-state-change code to execute, respectively.
  
  The \smg{<current-state>} may be ``\smg{*}'' to indicate that the
  transaction is a default transaction and applies to all states.
  
\item[\hypertarget{TRANSplus}{\smg{TRANS+}}] --- A special form of the
  \TRANS directive.  This directive indicates that the associated
  transaction information is defining ``group'' code that is performed
  in addition to the normal transaction.  There must only be one
  \TRANS for a current-state/event/destination-state transaction, but
  there may be zero or more \smg{TRANS+} (in addition to the \TRANS)
  for that same transaction that specify additional code.  This
  directive is especially useful for specifying default code that is
  associated with an event that occurs in any state (using a wildcard
  specifier for the current-state as described below).
  
  Another way of describing the difference between \TRANS and
  \smg{TRANS+} is that a multiply defined state transition error will
  only be detected for \TRANS directives.
  
  The \smg{<new-state>}, \smg{<pre-code-tag>}, \smg{<post-code-tag>},
  and \smg{<current-state>} fields may have the special values
  described in the \TRANS directive and have the same effect.

\item[\hypertarget{TRANSeq}{\smg{TRANS=}}] --- Another special form of
  the \TRANS directive.  This form is used to indicate the normal or
  expected transition in the current state (i.e. the primary code
  path).  When the event described in this \smg{TRANS=} directive has
  corresponding entry point and preprocessing code directives, the SMG
  preprocessor outputs specific code testing for and implementing this
  transaction before passing into the more general transaction
  processing routine, thereby increasing the efficiency and
  performance of the ``common path'' code.
  
  The \smg{<new-state>}, \smg{<pre-code-tag>}, \smg{<post-code-tag>},
  and \smg{<current-state>} fields may have the special values
  described in the \TRANS directive and have the same effect.

\item[\smg{##}] --- When placed at the start of a line, this specifies
  that the line is an SMG comment; SMG comments will not be reproduced
  in the output C code.
  
\item[\hypertarget{CODE}{\smg{CODE}}] --- Specifies a line of code and its associated tag.
  All code referenced by \EVENT and \TRANS directives is
  referred to by an associated code tag; the \smg{CODE} directive
  specifies the actual C code that is associated with that tag.
  
  Within the C code associated with the \smg{CODE} tag, special
  keywords will be recognized and appropriate substitutions will be
  made in the output C code generated.  All code-internal keywords are
  of the form ``\code{_/xxx}'' and may be one of the following:

\begin{description}
  
\item[\smg{_/OBJ}] --- Substitute the name of the \SMOBJ-typed
  variable.
  
\item[\smg{_/EVT}] --- Substitute the name of the \SMEVT-typed
  variable
  
\item[\smg{_/NAME}] --- Substitute the name of the current state machine
  being defined.
  
\item[\smg{_/<STATE>}] --- Specifies that code should be inserted to set
  the state to the specified value.  This is used for situations where
  the destination state cannot be determined solely from the current
  state and event.  The code which is specified for this type of
  \TRANS operation must programmatically make the determination
  of the appropriate target state and then set that state using the
  ``\code{_/<STATE>}'' keyword.
  
\item[\smg{_/<EVENT>}] --- Specifies that the associated event should be
  delivered to the state machine.  This event is delivered
  {\em IMMEDIATELY} to the state machine, and is comparable to a
  recursive invocation of the state machine.  It is possible to defer
  the delivery of these events to the end of the handling for the
  current event using a command-line flag, in which case they will be
  delivered in the order they were generated after all tagged code
  associated with the event has been executed.
  
  If it is desireable to queue events to a state machine instead, for
  both external event deliveries and internally generated events, a
  separate mechanism must be provided by the user to implement this
  queueing.  These directives should be used {\em CAREFULLY}.

\end{description}

\item[\hypertarget{CODEbegin}{\smg{CODE_\{}}] --- Specifies the start
  of a multi-line section of code and its associated tag.  This
  directive is special in that the code associated with this directive
  spans multiple lines, up to the closure directive.  Other than the
  multi-line aspect, this directive is exactly like the \smg{CODE}
  directive, including the keyword substitution activities.
    
\item[\hypertarget{CODEend}{\smg{CODE_\}}}] --- Specifies the end of a
  multi-line section of code started by the \CODEbegin directive.
  
\item[\hypertarget{PROMELAbegin}{\smg{PROMELA_\{}}] --- Specifies the
  start of a multi-line Promela code section.  Promela is a modeling
  language that may be used with the Spin utility to model and
  validate a state machine.  Promela code is output to the \file{.pml}
  file rather than the \file{.c} and \file{.h} files for corresponding
  \CODE segments.
  
  Special code tags of \smg{HEADER} and \smg{INIT} may be used to
  specify Promela code that should be output to the header or the end
  (init) of the Promela file.  One common initialization function is
  to declare the Promela channel used to deliver events to the state
  machine and then run the state machine, passing that channel.  The
  channel messages should be declared as follows:
  
%  \ \ \listcode{chan send_events = [0] of \{ mtype, mtype \};}
  
  Where the first mtype should be either \smg{INITIALIZE_SM} or
  \SM{EVENT} as appropriate, and the second mtype should be the
  initial state or the event name, respective to the first mtype.  The
  depth of the channel should always be zero to accurately model the
  immediate delivery mechanism of the state machine; queued event
  delivery should use a separate queueing process rather than simply
  expanding the channel depth to provide the proper semantics.
  
  See the output of the examples for more information.
  
\item[\hypertarget{PROMELAend}{\smg{PROMELA_\}}}] --- Specifies the
  end of a multi-line Promela code section.

\end{description}



\subsection{SMG Wildcards}

When specifying the arguments for the SMG directives described above,
there are special wildcards that are recognized by the SMG
preprocessor that may be used in place of a more customary value for
that argument:

\begin{description}
  
\item[\smg{*}] --- Wildcard argument.  This is typically useable in
  place of a state name in a \TRANS directive, indicating that any and
  all states apply for that transition.  This can be useful in
  specifying the ``default'' transition for an event, and may be
  overridden for specific states by subsequent \TRANS directives that
  do not use the wildcard.
  
\item[\smg{--}] --- No-action argument.  This indicates that nothing
  should occur for the corresponding argument.  For example, when used
  in place of the \smg{new_state} name in the \TRANS statement,
  this indicates that the event does not change the current state.
  When used in place of a code tag, it indicates that there is no code
  to be executed.

\end{description}

\subsection{SMG Syntax Version}

Currently the SMG directives syntax is unchanged from the original
publicly distributed version.  In the future, the SMG directives
syntax might need to change to accomodate additions, changes, and
removals from the current syntax.  To allow \file{.sm} files to be
written to conform to multiple syntax forms and/or validate the needed
syntax to interpret the current file, SMG automatically generates one
or more context definitions describing the syntax interpretation used
by that \code{smg} utility.

This automatically generated context definition may be checked with
the \SMIF directive, even though no \SMDEF directive explicitly
defined that context.

Syntax extensions and additions are represented by additional
automatically generated contexts defined simultaneously with the
original syntax definition where the new syntax is backward-compatible
with the original syntax.  Syntactic changes incompatible with
previous versions will be represented by a newly unique automatic
context definition and the lack of previous definitions.

Also note that the syntax versioning supported in this manner is
independent of the SMG utility version; multiple SMG utility versions
might support identical syntax forms.

\begin{tabular}{l|p{1.8in}}
{\bf Context} & {\bf Description} \\\hline
\smg{SMG_SYNTAX_A} & Original SMG syntax \\
\end{tabular}


\section{SMG Specification Requirements}

In addition to the SMG directives described above, there are a few
requirements for successfully integrating the generated state machine
C code into the rest of the software module:

\begin{enumerate}
  
\item The \SMOBJ structure must contain a field (\code{sm_state}) that
  can be used by SMG to maintain the current state of the state
  machine.  This field should NEVER be directly accessed by the C
  code.

   
\item When a run-time error occurs, the generated state machine C code
  will call an error function that must be supplied by the user-supplied
  C code; this is so that the error handling activity can be handled
  in a manner appropriate to the current implementation.
  
  The declaration for the error function to be provided by the
  user-supplied C code is defined as the
  \code{<SM_NAME>_State_Machine_Error} function in
  Figure~\ref{fig:interface}.
  
  User-supplied code may also call the error function, but if the
  \code{err_id} used matches the errors defined in
  Table~\ref{table:errs} then the \code{errtext} and additional
  parameters must additionally match.
  
  In the definition referenced, \code{<SM_NAME>} is replaced with the
  name of the state machine associated with this error (allowing
  separate error handlers for each state machine defined) and
  \code{<SM_OBJ>} and \code{<SM_EVT>} are replaced by the
  corresponding C type specifications from the similarly named SMG
  directives.
  
  The \code{errtext} describes the error and may be followed by
  arguments to be used in \code{printf}-style format codes.
  
  The \code{err_id} is an identifier value associated with this error.
  The \code{errtext} and the type and sequence of arguments for a
  specific \code{err_id} will never change, so error routines are free
  to key on the \code{err_id} value to perform specific actions as
  defined in Table~\ref{table:errs}.

\begin{figure*}[tb]
\begin{verbatim}
void <SM_NAME>_State_Machine_Error(<SM_OBJ> _sm_obj,
                                   <SM_EVT> _sm_evt,
                                   int   err_id,
                                   char *errtext, ...);

void <SM_NAME>_State_Machine_Event(sm_obj, sm_evt, event_code);

void <SM_NAME>_State_Machine_Init(sm_obj, initial_state);
\end{verbatim}
\caption{State Machine C-code Interface Declarations}
\label{fig:interface}
\end{figure*}

\begin{table*}[tb]
\caption{State Machine Error function \code{err_id} and \code{errtext} values}
\label{table:errs}
\begin{tabular}{|c|l|}
\hline
{\bf \code{err_id}} & {\bf \code{errtext}} \\ \hline \hline
0 & {\it unused} \\ \hline
1 & \code{Undefined State Transition (State <#>=<N>: <D>), (Event <#>=<N>: <D>)} \\ \hline
2 & \code{Invalid STATE\!\! (<#>=<N>: <D>)} \\ \hline
3 & \code{Invalid STATE/EVENT\!\! (State <#>=<N>: <D>) (Event <#>=<N>: <D>)} \\
\hline
 & \code{<#>} is the numeric value for the item in question \\
 & \code{<N>} is the corresponding name string \\
 & \code{<D>} is the corresponding description string \\
\end{tabular}
\end{table*}

\item When C code within the user-supplied software wishes to deliver
  an event to the state machine and therefore activate it to process
  that event to completion, it should call the
  \code{<SM_NAME>_State_Machine_Event} function as defined in
  Figure~\ref{fig:interface}.
  
  In the definition referenced, \code{<SM_NAME>} is replaced with the
  name of the state machine to which the event is to be delivered.
  The \code{sm_obj} and \code{sm_evt} arguments must be the
  appropriate entities with types defined by the \SMOBJ and \SMEVT SMG
  directives, respectively.  The \code{event_code} must be of type
  \code{<SM_NAME>_event_t}, where that type (and the actual event
  codes) are defined in the include header file output by the smg
  preprocessor.
  
\item For state machines which are not initialized at instantiation
  time (i.e. which do not contain an \INITSTATE directive) the state
  machine must be explicitly initialized before any events are
  delivered to the state machine.  A state machine must be explicitly
  initialized by calling the \code{<SM_NAME>_State_Machine_Init}
  function as defined in Figure~\ref{fig:interface}.
  
\item The \file{.sm} file should contain a ``\incode{#include
    "<FILE>_smdefs.h"} C statement, where \code{<FILE>} is the same as
  the \file{.sm} input filename (without the \file{.sm} extension).
  This header file is automatically generated by SMG and will define
  the states, events, and various SMG entry points in C-syntax code.
\label{item:smdef}
  
  This header file {\em must} be included into the \file{.sm} file at
  a minimum, and may be included into other \file{.sm} or
  \file{.c}/\file{.h} files in the module as needed.  Its inclusion
  must be explicit by the developer to insure that the inclusion
  occurs at the right point in the code (e.g. following any other
  definitions needed by the SM definitions, but prior to actual usage
  of those definitions in the user-supplied code).
  
\item The \file{_smdefs.h} inclusion in requirement \ref{item:smdef}
  must be preceeded by any type definitions required by code
  declarations in the \file{_smdefs.h} file, including the \SMOBJ and
  \SMEVT types.
  
\item The SMG code assumes that all external calls to
  \code{<SM_NAME>_State_Machine_Event} are externally synchronized and
  otherwise protected against multi-thread re-entrancy.
  
\item Code segments are output in the order in which they are
  specified for \TRANSplus code.  All pre-state \TRANSplus code
  preceeds normal pre-state code, and all post-state \TRANSplus code
  follows normal post-state code.

\end{enumerate}


\section{SMG Usage}

The SMG preprocessor is invoked from the command line prior to the C
file compilation.  It will produce a number of output files: a C code
file (\file{.c}), a header include file (\file{.h}), a GraphViz dot
input file (\file{.dot}), and a GraphViz output file in PostScript,
GIF, MIF (Framemaker), or HTML imagemap format.

The SMG preprocessor may be invoked with no arguments or with the
``\code{-h}'' flag to obtain explicit usage information.


\subsection{SMG Command Line Arguments}

The following describes the command line arguments for SMG in more
explicit detail than than provided by run-time help.

\begin{description}
  
\item[\code{-h}] Displays the usage/help information.  Usually
  displayed on a command-line parsing error as well.
  
\item[\code{-i}] Passes all generated \file{.c} and \file{.h} files
  through the indent program to improve readability.  The current user
  environment provides the appropriate input to indent; no indent
  parameters are passed on the command line used to invoke indent.
  The indent application must be present in the current \code{PATH}.
  Note that some indent operations may cause the line numbers reported
  during compilation to be skewed; in particular, the \code{-cdb} (GNU
  indent) option is evil.
  
\item[\code{-v}] Verbose output.  Enables various informational and
  progress messages.  If verbose output is not enabled, the SMG
  preprocessor will not generate any messages to stdout.
         
\item[\code{-D}] Defer tagged code event generation.  As described
  above, a \TRANS directive may specify one or more tags identifying
  code that is to be generated for handling that event, and that code
  may contain keywords of the type ``\incode{_#<event>}'' or
  ``\incode{_/<event>}'' where \code{<event>} is a valid event for
  this state machine.  Normally, the keyword will be directly replaced
  with appropriate code to generate the specified event, but if the
  \code{-D} flag is used, the code to generate the event will be
  placed at the end of the \TRANS code segments.
  
\item[\code{-N}] Nested switch statements for state machine handling.
  When an event occurs, there are two elements which are used to
  determine the appropriate handling of that event: the event code
  itself and the current state.  By default, these two values are
  combined and a singe C switch/case statement is generated to
  dispatch to the appropriate handling.  When the \code{-N} flag is
  utilized, one C switch/case statement is used to scan for the
  current state, and then within the case handling for that state,
  another C switch/case statement is used to scan for the
  \code{event_code}.
  
\item[\code{-T}] Trace code output.  When this flag is specified,
  state machine trace operations are embedded in the generated code.
  This is typically used for debugging and may or may not be specified
  for ``production'' versions of the code.  More information on
  tracing is provided in Section~\ref{section:tracing}.
         
\item[\code{-e}] Enum declarations.  By default, the
  \file{XXX_smdefs.h} file generated uses \incode{#define} statements
  for states and events.  When the \code{-e} flag is specified,
  \code{enum} statements are generated instead.
         
\item[\code{-b}] Bounds checking.  Adds code to the state machine
  event handler to validate the event value for validity as an actual
  event code.
  
\item[\code{-E}] Embedded code.  This causes SMG to output code
  suitable for use in an embedded environment.  This code will not
  contain the functions to translate a state or event to its
  corresponding string name or string description, and calls to the
  error function will pass an empty string as the errtext argument.
  Also note that it is improper to use tracing (\code{-T}) with
  embeddable output; the result will not compile because the tracing
  code requires the suppressed translation routines.
         
\item[\code{-l}] Suppress \incode{#line} directives.  By default, the
  generated code contains ``\incode{#line}'' C precompiler directives
  that reference the input SMG file.  This greatly aids in debugging
  as the compilation stage will usually reference the correct location
  in the input SMG file rather than the generated (and therefore
  somewhat unfamiliar) C file.  By specifying the \code{-l} flag,
  these directives are suppressed and any C compiler messages will
  refer to the \file{.c} file instead; this may be useful for
  debugging errors that are not apparent in the SMG input file
  directly.
  
\item[\code{-G}] GIF diagram output format.  Specifies that the state
  machine diagram that is generated is to be encoded in GIF format.
         
\item[\code{-P}] Postscript diagram output format.  Specifies that the
  state machine diagram that is generated is to be encoded in
  Postscript format.  This is the default.
         
\item[\code{-M}] MIF diagram output format.  Specifies that the state
  machine diagram that is generated is to be encoded in MIF format
  (Adobe FrameMaker's Interchange Format).
         
\item[\code{-W}] Web imap diagram output format.  Specifes that the
  state machine diagram that is generated is to be encoded as an html
  image map suitable for use on a Web page.

\end{description}


\subsection{Support Functions}

The SMG utility will automatically generate a set of support functions
in addition to the primary \code{<name>_State_Machine_Init} and
\code{<name>_State_Machine_Event} functions.  These support functions
may be used by the user's event-specific code or other code to obtain
user-readable names and descriptions of the States and Events defined
in the \file{.sm} file.

The following support functions (defined in
Figure~\ref{fig:support-functions}) are generated automatically by
SMG:

\begin{figure*}
\begin{verbatim}
char *<name>_State_Name(<name>_state_t state);
char *<name>_State_Desc(<name>_state_t state);
char *<name>_Event_Name(<name>_event_t event);
char *<name>_Event_Desc(<name)_event_t event);
\end{verbatim}
\caption{SMG Auto-generated Support Functions}\label{fig:support-functions}
\end{figure*}

\begin{description}
\item[\code{<name>_State_Name}] This function converts its state value
  argument into a character string name for that state.
\item[\code{<name>_State_Desc}] This function converts its state value
  argument into a character string description for that state.
\item[\code{<name>_Event_Name}] This function converts its event value
  argument into a character string name for that event.
\item[\code{<name>_Event_Desc}] This function converts its event value
  argument into a character string description for that event.
\end{description}

Note that the generation of these support functions is suppressed when
the \code{-E} argument is used to request embeddable output code.


\subsection{Run-Time Tracing}
\label{section:tracing}

When the SMG utility is invoked with the \code{-T} argument, the
generated SMG code contains run-time tracing output.  By default, this
tracing output causes information to be printed to stderr regarding
the various events and state changes that occur when the SMG-generated
state machine operates.


%%% Local Variables: 
%%% mode: latex
%%% TeX-master: "smg_guide"
%%% End: 


Some environments do not provide fprintf access to a stderr output
stream, and other situations might desire custom control over the
tracing output.  These scenarios can be handled by overriding the
default tracing code generated.  The overrides are specified in the
form of a set of \code{#define} statements in the \file{.sm} file.
These \code{#define} statements should appear fairly early in the
\file{.sm} file, and must appear before the inclusion of the
\file{xxx_smdefs.h} file.

The following macros should be overridden with specific \code{#define}
statements to modify the tracing behavior:

\begin{description}
\item[\code{SM_TRACE}] This macro should be defined to signal an
  override of the default tracing functionality generated by SMG.
  This macro does not need to be set to a specific value or
  translation: it simply needs to be defined.
\item[\code{SM_TRACE_INIT}] This macro is called when the state
  machine is initialized.  This macro can be used to trace the
  initialization event and the initial state of the state machine.
  The parameters for this macro are:
  \begin{description}
  \item[\code{Obj}] The \code{_/OBJ} variable for the state machine
    context.  (Type: declared by the \SMOBJ directive).
  \item[\code{Evt}] The \code{_/EVT} variable for this event.  (Type:
    declared by the \SMEVT directive).
  \item[\code{SM_Name}] The \SMNAME of the state machine.  (Type:
    \code{char *}).
  \item[\code{InitState}] The initial state value.  The
    \code{<name>_State_Name} and \code{<name>_State_Desc} support
    functions may be called to translate the state value into an
    identifying string and corresponding description if desired.
    (Type: \code{<name>_state_t})
  \end{description}
\item[\code{SM_TRACE_EVENT}] This macro is called when an event
  occurs.  This macro should identify the event which occurred and the
  new state of the State Machine.  The parameters for this macro are:
  \begin{description}
  \item[\code{Obj}] The \code{_/OBJ} variable for the state machine
    context.  (Type: declared by the \SMOBJ directive).
  \item[\code{Evt}] The \code{_/EVT} variable for this event.  (Type:
    declared by the \SMEVT directive).
  \item[\code{SMNAME}] The \SMNAME of the state machine.  (Type:
    \code{char *}).
  \item[\code{Event}] The event value.  The \code{<name>_Event_Name}
    and \code{<name>_Event_Desc} support functions may be called to
    translate the event value into an identifying string and
    corresponding description if desired.  The new state value is
    available from \code{_/OBJ->sm_state} and it may likewise be
    converted to readble strings by the corresponding support
    functions.  (Type: \code{<name>_event_t})
  \end{description}
\item[\code{SM_TRACE_EXP_EV}] This macro is called when an expected
  event occurs ({\it i.e.} an event transition declared with a
  \TRANSeq directive).  This macro is otherwise identical to the
  \code{SM_TRACE_EVENT} macro, including the arguments, but this macro
  may wish to differentiate the occurrence of an expected event from
  the occurence of a normal event in the generated trace output.
\end{description}


%%% Local Variables: 
%%% mode: latex
%%% TeX-master: "smg_guide"
%%% End: 
