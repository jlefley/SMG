\section{SMG Libraries}

An interesting and useful extension of SMG usage is as a new method
for defining interface libraries, especially for asynchronous
event--driven interfaces.  SMG features that make it very useful in
working with event--driven architectures.  Event--driven architectures
are often asynchronous by nature and lend themselves to state machine
management techniques, but SMG has the ability to specify entry points
and setup code for those asynchronous entry points as an SMG Library
that can be included into the applications SMG specification using the
\SMINCL directive.

\subsection{Event Interface Background}

For example, consider a hypothetical windowing environment called "Y".
Applications written for Y-windows must specify functions that are
called when certain events occur within their window.  The following
events must be handled by a Y-windows application:

\begin{itemize}
\item Mouse click, Button 1
\item Mouse click, Button 2
\item Keyboard key entered
\item Expose event (the window has become visible)
\item Iconify event (the window should be iconicized)
\end{itemize}

For a C development environment, there would be a Y-windows header file
that defined these events and an event vector structure that would have
to be initialized by the application:

\begin{verbatim}

typedef struct y_app_vtable {
      void *y_mouse_b1(int x, int y);
      // tricky: arguments switched:
      void *y_mouse_b2(int y, int x);
      void *y_key_entered(char keyval);
      void *y_expose(void);
      void *y_iconicize(void);
}  y_app_vtable_t;

int y_app_mainloop(void);

\end{verbatim}

The Y-windows application would then need to define functions to be
entered into the \code{y_app_vtable_t} to handle the various events.
The application would also need to call \code{y_app_mainloop} in its
main routine after initializing so that Y-windows could begin processing
events and passing them to the application's handling functions.

The inconvenience of this methodology is that the process of creating
the entry functions is a tedious, mechanical process that each Y-windows
application must perform.  This is exacerbated by the fact that the first
thing that each entry point must do is check the global state to see how
the event should be handled.  For example, \file{myapp.c} would contain
the following minimum code irrespective of the specifics of the application:

\begin{verbatim}

void myapp_mouse_b1(int x, int y) {
    switch (app_state) {
    case STATE_1:
          <some code>
          break;
    case STATE_2:
          <some code>
          break;
          :
    }
}

void myapp_mouse_b2(int y, int x) {
    switch (app_state) {
    case STATE_1:
          <some code>
          break;
    case STATE_2:
          <some code>
          break;
          :
    }
}

void myapp_key_entered(char keyval) {
      // same as above...
}

void myapp_exposed(void) {
      // same as above...
}

void myapp_iconicize(void) {
      // same as above...
}

\end{verbatim}


Instead of exposing the developer to the tedium of manually writing
this template an SMG Library can be provided for Y-windows. This also
removes the risk of entering it incorrectly, (eg. as \code{void
  myapp_mouse_b1(int y, int x)} which is syntactically but not
functionally correct).


\subsection{SMG Library Implementation}

An SMG Library is nothing more than an SMG file that should be \SMINCL
included into the main application and which defines the events and entry
points for those events.  This library can be developed once and re-used
by all applications that operate within that event-driven architecture.
The application developer is freed to focus on the actual functionality
of the application in response to the events rather than the mechanics
of program infrastructure.

A subset of the example above can be used to show how the introduction
of an SMG Library changes the development.  The initial introduction
of the SMG Library generated state machine would replace all the
switch statements in the example above into calls to the
\code{myapp_State_Machine_Event} function, thereby saving some of the
superfluous repetition, but also making the entry points mere shells:

\begin{verbatim}
void myapp_iconicize(void) {
    myapp_State_Machine_Event(&myapp_global,
                              <event_obj>,
                              Iconicize_E);
}
\end{verbatim}

Because the state machine function will vector the code to the proper
handling code, the entry point function needs to do little more than marshall the entry points arguments into the \SMEVT object and call the
SMG's state machine function.  


%%% Local Variables: 
%%% mode: latex
%%% TeX-master: "smg_guide"
%%% End: 
