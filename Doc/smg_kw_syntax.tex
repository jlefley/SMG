The syntax of SMG Directives is described in Figure~\ref{fig:syntax}.
In the figure, \smg{<xxx>} represents a user-supplied word, and
\smg{<xxx>...}  represents one or more user-supplied words.  Square
braces ('[' and ']') enclose optional portions of the directive and
are not actually present in the directive.  There are no restrictions
on \smg{<xxx>} words except those noted here and excepting whitespace.
Some \smg{<xxx>} words must be valid C identifiers:
\smg{<state-machine-name>}, \smg{<state-name>}, and
\smg{<event-name>}.

All SMG directives are line oriented, must be contained on a single
line, and must start with the SMG directive keyword at the first
character on the line.  The remainder of the line contains the
arguments and values for the SMG directive as whitespace separated
words, up to the end of the line; SMG directives do not support
line-continuation.

\begin{figure*}[p]
\begin{verbatim}

SM_NAME <state-machine-name>

SM_DESC <description>...

SM_OBJ  <C-type>...

SM_EVT  <C-type>...

SM_INCL <filename>

SM_DEF  <context-name>
SM_IF   <context-name>
SM_ELSE <context-name>
SM_END  <context-name>

STATE   <state-name> [ <description>... ]

INIT_STATE <state-name> [ <description>... ]

ST_DESC <description>...

EVENT <event-name> [ <entry-code-tag> <setup-code-tag> [ <description...> ] ]

EV_DESC <description>...

TRANS  <current-state> <event> <new-state> [ [<pre-code>] <post-code> ]
TRANS+ <current-state> <event> <new-state> [ [<pre-code>] <post-code> ]
TRANS= <current-state> <event> <new-state> [ [<pre-code>] <post-code> ]

## <comment>...

CODE    <code-tag> <code>...

CODE_{  <code-tag>
CODE_}

PROMELA_{      <code-tag>|HEADER|INIT
PROMELA_}

\end{verbatim}
\caption{SMG Directive Syntax}
\label{fig:syntax}
\end{figure*}


%%% Local Variables: 
%%% mode: latex
%%% TeX-master: "smg_guide"
%%% End: 
