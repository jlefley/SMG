
\subsection{File Example 1}

This example shows a simple example of a menu-driven file processing
application that has an underspecified state machine.  

The example is located in the \file{Examples/file_ex1} subdirectory of
the SMG distribution.  The following sequence of events may be used to
work with this example:

\begin{center}
\begin{picture}(60,70)(0,0)

\put(17,55){\href{Examples/file_ex1/file_ex1.sm}{\framebox(26,15){\shortstack[l]{...\\\code{C code}\\\code{SM directives}\\...}}}}
%\put(17,55){\framebox(26,15){\shortstack[l]{...\\\code{C code}\\\code{SM directives}\\...}}}
\put(46,62){\href{Examples/file_ex1/file_ex1.sm}{.sm}}

\put(30,55){\vector(0,-1){5}}

\put(31,46){\oval(13,7)[c]}
  \put(27,45){\href{Examples/file_ex1/file_ex1.log}{SMG}}
%  \put(30,40){\line(0,1){2}}
\put(30,42){\vector(-2,-1){11}}
  \put(0,32){\href{Examples/file_ex1/file_ex1_smdefs.h}{\file{xxx_smdefs.h}}}
\put(30,42){\vector(0,-1){6}}
  \put(28,32){\href{Examples/file_ex1/file_ex1.c}{\file{.c}}}
\put(30,42){\vector(2,-1){11}}
  \put(38,32){\href{Examples/file_ex1/file_ex1S.dot}{\file{.dot}}}
\put(30,42){\vector(4,-1){30}}
  \put(59,32){\href{Examples/file_ex1/file_ex1.pml}{\file{.pml}}}%52

\put(20,21){\oval(8,6)[c]}
    \put(17,20){\hyperlink{filex1smg}{CC}}
    \put(16,30){\vector(1,-3){2}}
    \put(30,30){\vector(-1,-1){7}}
\put(16,10){\file{.exe}}
    \put(20,18){\vector(0,-1){5}}

\put(43,21){\oval(18,6)[c]}
    \put(36,20){\hyperlink{GraphViz}{GraphViz}}
    \put(42,30){\vector(0,-1){6}}
\put(38,10){\href{Examples/file_ex1/file_ex1S.ps}{\file{.ps}}}
    \put(42,18){\vector(0,-1){5}}

\put(61,21){\oval(10,6)[c]}
    \put(58,20){\hyperlink{Spin}{Spin}}
    \put(61,30){\vector(0,-1){6}}
\put(56,7){\shortstack{Verification\\Report}}
    \put(61,18){\vector(0,-1){5}}
\end{picture}
\end{center}

\hypertarget{filex1smg}{\subsubsection{Building the Example}}

In the following example, the commands entered are shown in {\tt\it italics} and the output of the utilities is shown in {\tt this font}.
\begin{verbatim}
$ smg -viTP file_ex1
$ gcc -o file_ex1 file_ex1.c
\end{verbatim}

Once built, it may be run by executing:

\begin{verbatim}
$ file_ex1
\end{verbatim}

In addition to the executable, the source files (\file{file_ex1.c} and
\file{file_ex1_smdefs.h}) may be examined.  There is also a Promela
model generated (\file{file_ex1.pml}) and a graphical representation
of the state machine
(\href{Examples/file_ex1/file_ex1S.ps}{\file{file_ex1S.ps}}) based on
GraphViz directives (\file{file_ex1S.dot}).

Examining the graphical representation, it can easily be observed that
the state machine is underspecified by the presence of the
\smg{UNDEFINED_TRANSITION_RESULT} state.  This can be confirmed by
performing a sequence of menu operations with the executable that lead
to this error state and observing the output dynamically.

%%\newsavebox{\fileex1fig}
%%\sbox{\fileex1fig}{\includegraphics{file_ex1S.ps}}
%\begin{figure*}
%%\begin{center}\usebox{\fileex1fig}\end{center}
%\begin{center}\includegraphics{file_ex1S}\end{center}
%\end{figure*}


\subsection{Unmorse Example}

This example shows the use of an SMG-generated state machine for
translating from Morse Code back to plain-text.

The example is located in the \file{Examples/unmorse} subdirectory of
the SMG distribution.  The following sequence of events may be used to
work with this example:

\begin{center}
\begin{picture}(60,70)(0,0)

\put(17,55){\href{Examples/unmorse/unmorse.sm}{\framebox(26,15){\shortstack[l]{...\\\code{C code}\\\code{SM directives}\\...}}}}
%\put(17,55){\framebox(26,15){\shortstack[l]{...\\\code{C code}\\\code{SM directives}\\...}}}
\put(46,62){\href{Examples/unmorse/unmorse.sm}{.sm}}

\put(30,55){\vector(0,-1){5}}

\put(31,46){\oval(13,7)[c]}
  \put(27,45){\href{Examples/unmorse/unmorse.log}{SMG}}
%  \put(30,40){\line(0,1){2}}
\put(30,42){\vector(-2,-1){11}}
  \put(0,32){\href{Examples/unmorse/unmorse_smdefs.h}{\file{xxx_smdefs.h}}}
\put(30,42){\vector(0,-1){6}}
  \put(28,32){\href{Examples/unmorse/unmorse.c}{\file{.c}}}
\put(30,42){\vector(2,-1){11}}
  \put(38,32){\href{Examples/unmorse/unmorseS.dot}{\file{.dot}}}
\put(30,42){\vector(4,-1){30}}
  \put(59,32){\href{Examples/unmorse/unmorse.pml}{\file{.pml}}}%52

\put(20,21){\oval(8,6)[c]}
    \put(17,20){\hyperlink{filex1smg}{CC}}
    \put(16,30){\vector(1,-3){2}}
    \put(30,30){\vector(-1,-1){7}}
\put(16,10){\file{.exe}}
    \put(20,18){\vector(0,-1){5}}

\put(43,21){\oval(18,6)[c]}
    \put(36,20){\hyperlink{GraphViz}{GraphViz}}
    \put(42,30){\vector(0,-1){6}}
\put(38,10){\href{Examples/unmorse/unmorseS.ps}{\file{.ps}}}
    \put(42,18){\vector(0,-1){5}}

\put(61,21){\oval(10,6)[c]}
    \put(58,20){\hyperlink{Spin}{Spin}}
    \put(61,30){\vector(0,-1){6}}
\put(56,7){\shortstack{Verification\\Report}}
    \put(61,18){\vector(0,-1){5}}
\end{picture}
\end{center}

\hypertarget{filex1smg}{\subsubsection{Building the Example}}

In the following example, the commands entered are shown in {\tt\it italics} and the output of the utilities is shown in {\tt this font}.
\begin{verbatim}
$ smg -viTP unmorse
$ gcc -o unmorse unmorse.c
\end{verbatim}

Once built, it may be run by executing:

\begin{verbatim}
$ unmorse '- . ... - -..-. .---- .-.-.-'
Translation: TEST/1.
$
\end{verbatim}

In addition to the executable, the source files (\file{unmorse.c} and
\file{unmorse_smdefs.h}) may be examined.  There is also a Promela
model generated (\file{unmorse.pml}) and a graphical representation
of the state machine
(\href{Examples/unmorse/unmorseS.ps}{\file{unmorseS.ps}}) based on
GraphViz directives (\file{unmorseS.dot}).


%%% Local Variables: 
%%% mode: latex
%%% TeX-master: "smg_guide"
%%% End: 
